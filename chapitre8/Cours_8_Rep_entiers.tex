\include{MeijeCours}

\newboolean{versionprof}  % Cette variable booléenne permet le contrôle de la compilation selon la version professeur (avec les preuves, commentaires pédagogiques etc...) si true ou selon la version élève (de ''base'') si false.
\setboolean{versionprof}{false}

% =============================================================================================
%                                                                                      Début du texte 
% =============================================================================================

\author{Informatique}  % Pour spécifier l'auteur qui sera affiché sous le titre
\title{Représentation des entiers}  % Titre 
\date{Année 2016-2017} %en utilisant \today on obtiendra la date courante lors de la production du document
\renewcommand{\thesection}{\Roman{section}}  % permet de numéroter les sections en chiffres romains
\pagestyle{fancy}

\lhead{NSI}   %  Indiquer la classe ici  
\rhead{Représentation des entiers}		% Indiquer la date pour rendre le devoir ici


\fancyfoot[C]{\thepage} % bas de page : centre par exemple:  - \thepage\ -
\fancyfoot[L]{} % bas de page : gauche
\fancyfoot[R]{} % bas de page : droite
\newcommand{\p}[1]{ \left( #1 \right)}   %parenthèse
\newcommand\abs[1]{|{#1}|}
\newcommand{\intent}[2]{[\![ #1 , #2 ]\!]} %intervalle fermé entier
\newcommand{\pe}[1]{ \left\lfloor #1 \right\rfloor} %partie entière crochets bas
\begin{document}
\rule[0pt]{0pt}{0pt}
\vspace{0.2cm}
%\maketitle  %produit le titre conformément aux instructions données par \author, \date ...etc
\begin{center}
{\LARGE\bf Représentation des entiers}
\end{center}
\thispagestyle{empty}
% ------------------------------------------------   Introduction   -------------------------------------------
\vskip 1cm
\renewcommand{\abstractname}{Introduction\hfill}

\section{{\'E}criture en base $b$ d'un entier}

$2021$ signifie dans notre écriture \textbf{décimale} $2\times 10^3 + 0 \times 10^2 + 2 \times 10^1 + 1 \times 10^0$. C'est ce que l'on appelle une écriture en
base $10$. Mais d'autres bases peuvent être utilisées. En informatique, on utilise, outre la base $10$, les bases $2$ (\textbf{binaire}) et $16$ (\textbf{hexadécimal})\footnote{On utilise alors A,B,C,D,E et F pour représenter les valeurs 10, 11, 12, 13, 14 et 15.}.
\bigskip

\begin{définition}{(\'Ecriture dans une base)}
Si $b$ est un entier supérieur ou égal à $2$, une \textbf{écriture en base $b$} d'un nombre $x \geq 1$ est de la forme
$$x = r_n b^n + r_{n-1} b^{n-1} + \cdots + r_1 b + r_0 \text{ que l'on notera } \overline{r_n r_{n-1} \dots r_1 r_0}^{(b)}.$$ 
Les $r_i$ sont les \textbf{chiffres} et sont compris entre $0$ et $b-1$. $r_n$ étant le premier chiffre,
il est non nul par définition.

\end{définition}

\exercice{}
\begin{enumerate}
\item Jusifier que $17$ s'écrit $\overline{10001}^{(2)}$ en base 2.

\item Comment s'écrit $47$ en base 5 ? 


\item Que vaut $\overline{131}^{ (6)}$ ?

\item Que vaut $\overline{B7}^{ (16)}$ ?

\end{enumerate}


\lignes{12}

\entrainement
\begin{enumerate}

\item Comment s'écrit $47$ en base 4 ? 

\item Comment s'écrit $255$ en base 16 ?


\item Que vaut $\overline{88}^{ (8)}$ ?

\end{enumerate}
\smallskip
\vfill
\eject

Pour déterminer l'écriture d'un entier en base $b$, on peut utiliser deux algorithmes, suivant que l'on commence à calculer le premier chiffre en partant de la gauche ou le dernier.
\exercice{: De la gauche vers la droite}

\medskip

\'Ecrivez $181$ en binaire :


\begin{center}
\includegraphics[width=8cm]{binaire-grille}
\end{center}

\lignes{3}
\medskip


L'inconvénient de cette méthode est qu'il faut déjà calculer les puissances de $2$ mais elle est assez commode à la main pour de petits nombres. Par contre, informatiquement, on utilisera plutôt l'algorithme suivant (très simple à coder et efficace).

\bigskip
\exercice{: De la droite vers la gauche}
\medskip

On remarque que si $x = r_n b^n + r_{n-1} b^{n-1} + \cdots + r_1 b + r_0$ alors $x = b \p{r_n b^{n-1} + r_{n-1} b^{n-2} + \cdots + r_1 } + r_0$. Comme $0 \leq r_0 < b$, $r_0$ est le reste de la division euclidienne de $x$ par $b$. Ainsi, pour calculer $r_0$, on fait cette division euclidienne sous la forme $x=bq_0+r_0$ puis on itère sur $q_0$ ce qui donne $q_0 = b q_1 + r_1$ donc $x = b(bq_1+r_1)+r_0 = q_1 b^2 + r_1 b + r_0$ et on continue sur $q_1$ \dots 

Déterminer l'écriture binaire de $77$.

\begin{center}
	\includegraphics[width=8cm]{binaire-grille}
\end{center}

\lignes{3}


\entrainement{}

En utilisant successivement les deux méthodes vues dans les exercices précédent, écrire 205 en binaire.
\begin{center}
	---------------------
\end{center}


Il est souvent important de savoir combien de chiffres seront nécessaires pour écrire un nombre en binaire. Pour cela, on a besoin de la fonction $\log_2$.




\smallskip

\begin{propriété}{}
Il existe une fonction mathématique $\log_2$ qui vérifient les propriétés suivantes : 

\begin{itemize}
	\item $\log_2$ est croissante sur $]0, +\infty[$
	\item $\log_2(2) = 1$
	\item $\log_2(a\times b)=\log_2(a)+\log_2(b)$
	\item $\log_2(a/ b)=\log_2(a)-\log_2(b)$
	\item $\log_2(a^n) = n\log_2(a)$
\end{itemize}

\end{propriété}
On peux alors énoncer la propriété :

\begin{propriété}{}
Le nombre de chiffres de l'écriture de $x$ en base $2$ est $\pe{ \log_2(x)}+1$.
\end{propriété}

Remarque : $\pe{a}$ désigne la partie entière du nombre a. C'est le plus grand entier relatif inférieure ou égale à $a$. Par exemple : $\pe{3,26} = 3$, $\pe{-7} = -7$ et $\pe{-2,3} = -3$.

{\it Preuve: } Admis
\exercice{}
On donne $10^3 < 1024 = 2^{10}$. En déduire une majoration du nombre de chiffres binaires nécessaires pour écrire $10^{100}$.

\lignes{3}

\exercice{}

Les nombres $a$ et $b$ sont respectivement représentés avec $\alpha$ et $\beta$ chiffres en binaire. Donner le nombre de chiffres nécessaires pour représenter $a+b$ et $a\times b$.

\lignes{4}

\entrainement{}

Le nombres $a$ est représenté avec $\alpha$ chiffres en binaire. Donner le nombre de chiffres nécessaires pour représenter $a^n$.
\begin{center}
	------------------
\end{center}
\exercice{}

\'Ecrire une fonction \verb!base(n,b)! qui renvoie une liste donnant l'écriture en base $b \leqslant 10$ de $n$.


\begin{lstlisting}
>>> base(77, 2)
[1, 0, 0, 1, 1, 0, 1]
\end{lstlisting}

\lignes{7}
\entrainement{}

Même consigne qu'à l'exercice précédent mais en fixant le nombre de chiffres de l'écriture :
\begin{lstlisting}
>>> base_complet(77, 2, 8)
[0, 1, 0, 0, 1, 1, 0, 1]
\end{lstlisting}
\bigskip

\exercice{}

Même consigne qu'à l'exercice 6 mais renvoyer une chaine. On pourra avoir $b>10$.

\begin{lstlisting}
>>> base_chaine(255, 16)
'FF'
\end{lstlisting}
\lignes{12}


\entrainement{}

Même consigne qu'à l'exercice précédent mais en fixant le nombre de chiffres de l'écriture :
\begin{center}
	---------------------
\end{center}

Pour réaliser l'opération inverse, on utilise 
 l'\textbf{algorithme de Hörner} :\\

Imaginons que l'on dispose séquentiellement des chiffres d'un grand nombre. Par exemple 4, puis 5, puis 0, puis 7, puis 3, enfin 5. Comment construire pas à pas le nombre correspondant ?

\lignes{3}

\exercice

En déduire une fonction \verb!valeur(L, b):! qui calcule le nombre représenté par la liste de chiffres \verb!L! dans la base $b$.


\begin{lstlisting}
>>>valeur([1, 0, 1, 1, 0], 2)
22
\end{lstlisting}

\lignes{8}



\section{Une première approche pour les négatifs}

Les mémoires d'ordinateur sont constituées de "cases" prenant la valeur $0$ ou la valeur $1$ (il s'agit d'un circuit électronique ouvert ou fermé). Une telle case est appelée un \textbf{bit} (binary digit = chiffre binaire). 

\smallskip

Les ordinateurs travaillent en regroupant les bits par paquets. Un paquet de $8$ bits est appelé \textbf{octet} (byte).

\begin{center}
\includegraphics[width=8cm]{bits-octets}
\end{center}

Ainsi, pour stocker un entier, il est naturel d'utiliser sa représentation binaire. L'exemple ci-dessus est donc le stockage en mémoire du nombre $181$.

\medskip

Si l'on travaille sur un octet, on peut donc représenter les entiers entre $0$ et $1+2+\dots+2^7=2^8-1=255$.

D'une manière générale, si les nombres sont codés sur $n$ bits, on pourra représenter les entiers entre $0$ et $2^n-1$.

\medskip
Mais il faut évidemment pouvoir travailler avec des entiers relatifs. Une première idée consiste à réserver le premier bit (celui le plus à gauche, dit "bit de poids fort") au signe de l'entier (par exemple $0$ pour positif et $1$ pour négatif) et les $n-1$ autres bits à la valeur absolue. Pour $n=8$, on peut alors représenter les entiers entre $-127$ et $127$ :


\medskip


\begin{center}
\begin{tabular}{ccc}
\includegraphics[width=8cm]{binaire-positif} & & \includegraphics[width=8cm]{binaire-negatif} \\
\end{tabular}
\end{center}

Cette représentation des entiers pose plusieurs problèmes. Par exemple, le nombre $0$ admet deux représentations différentes : $00000000$ et $10000000$ et il y a problème pour l'addition si l'on applique l'algorithme usuel (bit par bit de la droite vers la gauche avec propagation de retenue) :

\begin{center}
\includegraphics[width=10cm]{addition-binaire}
\end{center}

On préfère donc utiliser une autre représentation :

\section{La notation en complément à $2$}

Commençons par regarder le cas particulier des entiers codés sur un octet. On décide de représenter les entiers entre $-128$ et $127$ (au lieu de $0$ à $255$) de la manière suivante :

\begin{itemize}

\item si $x \in \intent{0}{127}$ alors $x$ est codé par son écriture binaire sur un octet.

\item si $x \in \intent{-128}{-1}$ alors $x$ est codé par l'écriture binaire de $x+256$ sur un octet.

\end{itemize}


\begin{center}
\includegraphics[width=12cm]{complement-a-deux-1.eps} 
\end{center}

Ainsi, on fait correspondre à chaque entier de $\intent{-128}{127}$ un entier de $\intent{0}{255}$ qui est ensuite codé sur $8$ bits par sa représentation binaire. On remarquera que $0$ admet alors une seule écriture : $00000000$.

 Voici par exemple la notation en complément à deux de $-43$ et $117$ :

\begin{lstlisting}
>>> base_complet(-43+256, 2, 8)
[1, 1, 0, 1, 0, 1, 0, 1]
>>> base_complet(117, 2, 8)
[0, 1, 1, 1, 0, 1, 0, 1]
\end{lstlisting}


On remarque que le signe d'un entier est immédiat à déterminer via sa notation en complément à deux : il est donné par le bit de poids fort. Le nombre est positif ou nul lorsque le bit de poids fort vaut $0$ et négatif sinon.

Nous pouvons maintenant généraliser à la notation en complément à deux sur $n$ bits. On représente les entiers entre $-2^{n-1}$ et $2^{n-1}-1$ de la manière suivante :

\begin{itemize}

\item si $x \in \intent{0}{2^{n-1}-1}$ alors $x$ est codé par son écriture binaire sur $n$ bits.

\item si $x \in \intent{-2^{n-1}}{-1}$ alors $x$ est codé par l'écriture binaire sur $n$ bits de $x+2^n$.

\end{itemize}






\bigskip
 Voici deux codes permettant de passer d'un entier à sa notation en complément à deux et réciproquement :


\begin{itemize}

\item Notation en complément à $2$ de $n$ sur $N$ bits :\\

\begin{lstlisting}  
def complement_a_deux(n, N):
    if n >= 0:
        return base_complet(n, 2, N)
    else:
        return base_complet(n+2**N, 2, N)
\end{lstlisting}

\medskip

\item Opération inverse :\\

\begin{lstlisting}[escapeinside=;;]
def valeur_cads(L):
    N = len(L)
    if L[0] =;\,;= 0:
        return valeur(L[1:], 2)
    else:
        return valeurle(L, 2) - 2**N
\end{lstlisting}
\end{itemize}

La notation en complément à deux peut également être vue de la manière suivante :

\begin{center}
La liste de bits $[r_{n-1} , \dots, r_0]$ code l'entier $-r_{n-1} 2^{n-1} + r_{n-2}2^{n-2} + \dots + r_0$
\end{center}
On voit bien alors que $r_{n-1}$ donne le signe de l'entier.

\exercice{}
On travaille sur $8$ bits. Quel entier est codé par :

\begin{itemize}
\item $01110011$ :
\item $11000001$ :
\end{itemize}



\exercice{}
En utilisant cette remarque, écrire une fonction permettant de passer de la notation en complément à deux à l'entier représenté par cette notation :


\lignes{12}

 \textbf{Dépassement de capacité} : supposons que l'on travaille avec une représentation sur $8$ bits. On peut donc représenter les entiers de $\intent{-128}{127}$. Mais si l'on additionne deux tels entiers, il se peut que le résultat sorte de cet intervalle. Il se produit alors un dépassement de capacité (overflow ou underflow).

\medskip

Si l'on veut additionner deux nombres, on  commence par déterminer leur notation en complément à deux puis on applique l'algorithme habituel d'addition bit à bit avec retenue et on revient à la notation décimale. Le faire pour $55$ et $-21$ puis $54$ et $81$ :

\medskip



\begin{center}
\begin{tabular}{cc}
\begin{minipage}{9cm}
\begin{center}
\includegraphics[width=7cm]{binaire-grille.eps} \\
\bigskip
\includegraphics[width=7cm]{binaire-grille.eps} \\
\bigskip
\includegraphics[width=7cm]{binaire-grille.eps} \\
\end{center}
\end{minipage}
&
\begin{minipage}{9cm}
\begin{center}
\includegraphics[width=7cm]{binaire-grille.eps} \\
\bigskip
\includegraphics[width=7cm]{binaire-grille.eps} \\
\bigskip
\includegraphics[width=7cm]{binaire-grille.eps} \\
\end{center}
\end{minipage}
\end{tabular}
\end{center}




En Python 2, les entiers ont une taille limitée et dès qu'il y a dépassement de cette taille, ils changent de type et deviennent des entiers longs. En Python 3, il y a un seul type d'entiers et ils ont une taille illimitée.

\smallskip

La limite de la taille des entiers en Python 2 dépend de la machine ($32$ bits ou $64$ bits).


\exercice{}
Expliquez le code suivant, exécuté en Python 2 :

\medskip


\begin{center}
\begin{tabular}{cc}
\begin{minipage}{6cm}
\begin{lstlisting}
>>> 2**30
1073741824
>>> 2**31-1
2147483647L
\end{lstlisting}
\end{minipage}
&
\begin{minipage}{11cm}
\begin{lstlisting}
>>> x = sum(2**k for k in range(31))
>>> x
2147483647
>>> x+1
2147483648L
\end{lstlisting}
\end{minipage}
\end{tabular}
\end{center}


\lignes{4}

\entrainement{}
On travaille sur $16$ bits avec la notation en complément à deux.
\begin{enumerate}
\item Quels sont les entiers que l'on peut représenter ?
\item Trouver la notation en complément à $2$ de $21213$ et $-15155$.
\item Quels sont les entiers représentés par $1000000000001111$ et $0101010101000001$ ?
\item Additionner ces deux nombres via leur représentation puis vérifier.
\end{enumerate}
\vskip 2cm

\begin{center}
	{\bf\large  Exercices supplémentaires}

\end{center}
\exercice{}

\'Ecrire en Python une fonction \verb+sommeBinaire(L, M)+ qui renvoie la liste des chiffres de l'écriture binaire de la somme des nombres représentés en binaire par \verb+L+ et \verb+M+. 



\exercice{}

\'Ecrire en Python une fonction \verb+diffBinaire(L, M)+ qui renvoie la liste des chiffres de l'écriture binaire de la différence des nombres représentés en binaire par \verb+L+ et \verb+M+. 

\exercice{}

\'Ecrire en Python une fonction \verb+prodBinaire(L, M)+ qui renvoie la liste des chiffres de l'écriture binaire du produit des nombres représentés en binaire par \verb+L+ et \verb+M+. 

\exercice{}

\'Ecrire une fonction \verb!complement(b)! qui calcule le complément à 2 d'un nombre binaire donné sous forme d'un tableau de taille quelconque contenant uniquement de 0 et des 1. La fonction renverra un tableau de même taille.

  \end{document}





